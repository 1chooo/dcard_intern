% Unicampletter_example.tex - an example latex file to illustrate Unicampletter.cls
%
% Template by Brian Wood (brian.wood@oregonstate.edu).  Please feel free to send suggestions for changes; this template/cls is not exactly elegantly done!
%

% Modified by Raúl Rojas (r264532@dac.unicamp.br) to fit the need of UNICAMP students.

\documentclass[11pt]{Unicampletter}
\usepackage[english]{babel}
\usepackage{fontspec} 
\usepackage{tikz} 
\usepackage{xcolor}
	\definecolor{scured}{RGB}{0,64,152}
% \usepackage{lipsum}
\usepackage{fancyhdr}
\usepackage{lastpage}
\usepackage{eso-pic}
% \documentclass{article}

% This section define the geometry of de document

\usepackage{geometry}
\geometry{
    letterpaper,
    layout=letterpaper,
    left=1.0in,
    right=0.75in,
    bottom=1cm,
    top=4cm,
    twoside,    
    showcrop
}

% This section is just a bunch of busywork so that the second and following pages read ``Page X of Y''
\pagestyle{fancy}
\fancyhf{}
\renewcommand{\headrulewidth}{0pt}
\renewcommand{\footrulewidth}{0pt}
\rhead{Page \thepage \hspace{1pt} of \pageref{LastPage}}
%
%
% Set custom font here.  OSU uses Cambria truetype font.  Comment this line out if you do not have a Cambria font (originally included with this template) installed; computer modern (or whatever your current default font is) will be substituted.
%
\setmainfont{[Cambria.ttf]}[BoldFont  = [CambriaBold.ttf], ItalicFont  = [CambriaItalic.ttf], BoldItalicFont = [CambriaBoldItalic.ttf] ]
%
% The material below is a whole big dang thing whose purpose is just to set up a fixed coordinate system for \tikz so that you can put the Department or School address in the upper right-hand side without it moving all around every time you change something in the page.  I think it works.
% Defining a new coordinate system for the page:
%
% --------------------------
% |(-1,1)    (0,1)    (1,1)|
% |                        |
% |(-1,0)    (0,0)    (1,0)|
% |                        |
% |(-1,-1)   (0,-1)  (1,-1)|
% --------------------------
\makeatletter
\def\parsecomma#1,#2\endparsecomma{\def\page@x{#1}\def\page@y{#2}}
\tikzdeclarecoordinatesystem{page}{
	\parsecomma#1\endparsecomma
	\pgfpointanchor{current page}{north east}
	% Save the upper right corner
	\pgf@xc=\pgf@x%
	\pgf@yc=\pgf@y%
	% save the lower left corner
	\pgfpointanchor{current page}{south west}
	\pgf@xb=\pgf@x%
	\pgf@yb=\pgf@y%
	% Transform to the correct placement
	\pgfmathparse{(\pgf@xc-\pgf@xb)/2.*\page@x+(\pgf@xc+\pgf@xb)/2.}
	\expandafter\pgf@x\expandafter=\pgfmathresult pt
	\pgfmathparse{(\pgf@yc-\pgf@yb)/2.*\page@y+(\pgf@yc+\pgf@yb)/2.}
	\expandafter\pgf@y\expandafter=\pgfmathresult pt
}
\makeatother


%%------------------------------------------------------------------------------%%
%

%%%%%%%%%%% Put Recommender(your mentor) Information Here %%%%%%%%%%%
%
\def\name{Hugo ChunHo Lin \\
	Machine Learning Intern \\
	ATM(major), CS(minor), AI(program), NCU \\
	Your interests \\
	lcho0127@g.ncu.edu.tw
}
%
% List your degree(s), licences, etc. here if you like.
%\def\What{, Your degrees, etc.} 
%
% Set the name of your Department or School here

% I honestly don't know why the negative spacing is necessary to get the alignment of the first line correct.  This must be a ``\tikz'' thing.
%%%%%%%%%%%%%%%%%%  School or Department %%%%%%%%%%%%%%%
\def\Where{\hspace{-1.2mm}\textbf{\color{scured}
		ATM(major), CS(minor), AI(program)\\ National Central University
}} 

%%%%%%%%%%%%  Additional Contact Information %%%%%%%%%%%
%
% Set your preferred primary contact address here.
\def\Address{No. 400, Albert Einstein Avenue} 
%
\def\CityZip{Campinas, São Paulo, Brazil 13083-852%\\
%	P.R.China
} 
%
% Set your  e-mail here
\def\Email{\textbf{\color{scured}e-mail}: fxxxx@unicamp.br}
%
% Set your preferred contact number here
% \def\TEL{\textbf{\color{scured}Tel/Fax}: +86 532 6678xxxx}
%
% Set your department or personal website here
% \def\URL{\textbf{\color{scured}URL}: {https://ouc.edu.cn}}
%

%%%%%%%%%%%%%%%%  Footer information  %%%%%%%%%%%%%%%%%%
%
%  The next line is for your college, used as a footer.  If you prefer not to have this, just comment out these lines in favor of the line labeled "[[Alternate]]" below
% \def\school{\small{
% 		OUC $\cdot$
% 		~Department of xxxxx $\cdot$
% 		~No. 238 Songling Road $\cdot$
% 		~Qingdao, China P.R} } 
% \def\school{~}  % [[Alternate]]
%

%%%%%%%%%%%%%%%%%%%%%  Signature line  %%%%%%%%%%%%%%%%%%%%%
%
% Set your signature line here.
% One can add a signature image in a PDF file using the following code; this requires a file called "signature_block.pdf" to be installed in the same folder as the .tex file.  The vertical spacing (\vspace) and the scaling will have to be adjusted to get things to look correct for your particular signature image. Alternatively, comment out the following line in favor of the one labeled "[[Alternate]]" if you want to sign a paper copy of the letter.
%
\signature{
	\vspace{-8mm}
	\name
}
%\signature{\name}  % [[Alternate]]
%%------------------------------------------------------------------------------%%



% This block sets up the address on the right-hand side of the header. 
%
% The following lines just compile the information you set up into the LaTex letter variable "address" for later use.
%
%The following command "clears out" the default address so that it can be better set using \tikz
\address{}

\def\newaddress{
	\Where\\ 
	\Address\\ 
	\CityZip\\ 
	\Email
}
%
%%%%%%%%%%%  DATE  %%%%%%%%%%%%%%%%%%%%%%%%%
% If you want a date different from the current date, comment out the next line in favor of the line labeled "[[Alternate]]".  The ``\vspace{10mm}'' just moves the date down a tiny bit so it doesn't interfere with the header.  This can be adjusted to your preference.
%
\date{\vspace{-6mm} \today}
% \date{\vspace{2mm} 20 September 2020}  %[[Alternate]]
%
%%%%%%%%%%% Set the subject here if there is one  %%%%
%\subject{Stuff} % optional subject line



%%-----------------------------------------------------------------------------------------%%
%
\begin{document}
	%
	%
	%%%%%%%%  The "To" address goes here.
	%
	\begin{letter}{
% 			\textit{journal name}, IOP Publishing (publishing company)
		%	IOP Publishing%\\ 
			%SomeTown, SomeState 					       				  		 ~~SomeZip
		    }   
% This line sets up the return address to the right-side of the OSU logo.  The location is set with absolute node addresses using ``\tikz''.  It can still be a bit fussy, and you may need to alter this a little to get things to look right.  The bit that changes the position are the numbers in parentheses ``at (14.2,2.7)''
%
\begin{tikzpicture}[remember picture,overlay,,every node/.style={anchor=center}]
\node[text width=6.3cm] at (page cs:0.5,0.8){\small \newaddress};
\end{tikzpicture} 

%%%%%%  The ``opening'' is just the methd of address you would like to use at the start of the letter.
%
\opening{Dear DCard commited,}
		%%%%%%%%%% Body of letter   %%%%%%%%%%%%%%

	I/We am/are enclosing here with a manuscript entitled “[Title of the article]” for publication in \textbf{[journal name]} for possible evaluation. The Corresponding author of this manuscript is [corresponding author] and contribution of the authors as mentioned below with their responsibility in the research.
		
		\begin{enumerate}
		    \item Name of corresponding author, corresponding author.
		    
		    \item Name of co-author (if necessary), co-author.
		    
		    \item Name of supervisor, supervisor.
		\end{enumerate}
		
    In this paper, I/we report on / show that [xxxxxx]. I/We believe that this manuscript is appropriate for publication by \textit{[journal name]} because it… [specific reference to the journal’s Aims and Scope].  
		
	\textbf{[Add all necessary disclaimers]}
		

    With the submission of this manuscript, i/we would like to undertake that:
	
	\begin{itemize}
	   \item All authors of this research paper have directly participated in the planning, execution, or analysis of this study.
	        
	   \item All authors of this paper have read and approved the final version submitted.
	        
	   \item The contents of this manuscript have not been copyrighted or published previously.
	        
	   \item The contents of this manuscript are not now under consideration for publication elsewhere.
	        
	   \item The contents of this manuscript will not be copyrighted, submitted, or published elsewhere, while acceptance by the Journal is under consideration.
	        
	   \item There are no directly related manuscripts or abstracts, published or unpublished, by any authors of this paper.
	        
	\end{itemize}
	    
	This research project was partially or fully sponsored by [Founding institutions].
	    
	    \vspace{10mm}
	    
		Thank you for your consideration of this manuscript. 
		
		\vspace{5mm}
	
	
  		% Some useful guidelines:
  		% https://www.indeed.com/career-advice/resumes-cover-letters/how-to-write-cover-letter-for-research-paper
  		
  		% https://designcenter.uiowa.edu/editing-services/manuscript-tips/writing-manuscript-cover-letter
  		
  		% https://apastyle.apa.org/style-grammar-guidelines/research-publication/cover-letters
		
		%%%%%%% ``closing'' sets the sign-off line.
		
		\closing{Sincerely,}
		
		% Comment out/in the lines below as necessary
		%\encl{If an enclosure is provided, let them know what it is.}
		
		%\ps{A postscript if that is a thing you do.}
		
		%\cc{Someone Who Cares (and is copied).}
		
	\end{letter}
	
\end{document}
%
%%-----------------------------------------------------------------------------------------%%